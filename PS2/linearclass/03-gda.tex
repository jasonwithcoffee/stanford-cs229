\item \subquestionpointswritten{4}
Recall that in GDA we model the joint distribution of $(x, y)$ by the following
equations:
%
\begin{eqnarray*}
	p(y) &=& \begin{cases}
	\phi & \mbox{if~} y = 1 \\
	1 - \phi & \mbox{if~} y = 0 \end{cases} \\
	p(x | y=0) &=& \frac{1}{(2\pi)^{\di/2} |\Sigma|^{1/2}}
		\exp\left(-\frac{1}{2}(x-\mu_{0})^T \Sigma^{-1} (x-\mu_{0})\right) \\
	p(x | y=1) &=& \frac{1}{(2\pi)^{\di/2} |\Sigma|^{1/2}}
		\exp\left(-\frac{1}{2}(x-\mu_1)^T \Sigma^{-1} (x-\mu_1) \right),
\end{eqnarray*}
%
where $\phi$, $\mu_0$, $\mu_1$, and $\Sigma$ are the parameters of our model.

Suppose we have already fit $\phi$, $\mu_0$, $\mu_1$, and $\Sigma$, and now
want to predict $y$ given a new point $x$. To show that GDA results in a
classifier that has a linear decision boundary, show the posterior distribution
can be written as
%
\begin{equation*}
	p(y = 1\mid x; \phi, \mu_0, \mu_1, \Sigma)
	= \frac{1}{1 + \exp(-(\theta^T x + \theta_0))},
\end{equation*}
%
where $\theta\in\Re^\di$ and $\theta_{0}\in\Re$ are appropriate functions of
$\phi$, $\Sigma$, $\mu_0$, and $\mu_1$.\\

{\bf BEGIN PROOF HERE}\\

For shorthand, we let $\mc{H} = \{\phi, \Sigma, \mu_{0}, \mu_1\}$ denote
the parameters for the problem.
Since the given formulae are conditioned on $y$, use Bayes rule to get:
\begin{align*}
	p(y =1| x ; & \mc{H})\\
	& = \frac {p(x|y=1; \mc{H}) p(y=1; \mc{H})} {p(x; \mc{H})}\\
	& = \frac {p(x|y=1; \mc{H}) p(y=1; \mc{H})}
		{p(x|y=1; \mc{H}) p(y=1; \mc{H}) + p(x|y={0}; \mc{H}) p(y={0};
		\mc{H})}\\\\
	&= \frac{\frac{1}{(2\pi)^{\di/2} |\Sigma|^{1/2}}
		\exp\left(-\frac{1}{2}(x-\mu_1)^T \Sigma^{-1} (x-\mu_1) \right) \phi} 
		{
		\frac{1}{(2\pi)^{\di/2} |\Sigma|^{1/2}}
		\exp\left(-\frac{1}{2}(x-\mu_1)^T \Sigma^{-1} (x-\mu_1) \right) \phi
		+ \frac{1}{(2\pi)^{\di/2} |\Sigma|^{1/2}}
		\exp\left(-\frac{1}{2}(x-\mu_{0})^T \Sigma^{-1} (x-\mu_{0})\right) (1-\phi)
		} \\
	&=  \frac{
		\exp\left(-\frac{1}{2}(x-\mu_1)^T \Sigma^{-1} (x-\mu_1) \right) \phi} 
		{
		\exp\left(-\frac{1}{2}(x-\mu_1)^T \Sigma^{-1} (x-\mu_1) \right) \phi
		+ 
		\exp\left(-\frac{1}{2}(x-\mu_{0})^T \Sigma^{-1} (x-\mu_{0})\right) (1-\phi)
		} \\
	&= \frac{
		1} 
		{
		1
		+ 
		\frac{\exp\left(-\frac{1}{2}(x-\mu_{0})^T \Sigma^{-1} (x-\mu_{0})\right) (1-\phi)}{\exp\left(-\frac{1}{2}(x-\mu_1)^T \Sigma^{-1} (x-\mu_1) \right) \phi}
		}\\
	&= \frac{1} {1+ 
		\frac{\exp\left(-\frac{1}{2}(x-\mu_{0})^T \Sigma^{-1} (x-\mu_{0})\right) \exp(\log(1-\phi))}{\exp\left(-\frac{1}{2}(x-\mu_1)^T \Sigma^{-1} (x-\mu_1) \right) \exp(\log(\phi))}
		}\\
	&= \frac{1} {1+ 
		\exp\left(-\frac{1}{2}(x-\mu_{0})^T \Sigma^{-1} (x-\mu_{0})
		+ \frac{1}{2}(x-\mu_1)^T \Sigma^{-1} (x-\mu_1) +\log((1-\phi)/\phi)
		\right) 
		}\\
	&= \frac{1} {1+ 
		\exp\left(-\frac{1}{2}(x^T-\mu_{0}^T) \Sigma^{-1} (x-\mu_{0})
		+ \frac{1}{2}(x^T-\mu_1^T) \Sigma^{-1} (x-\mu_1) +\log((1-\phi)/\phi)
		\right) 
		}\\
	&= \frac{1} {1+ 
		\exp\left(D +\log((1-\phi)/\phi)
		\right) 
		}\\
\end{align*}


\begin{align*}
D &= -\frac{1}{2}(x^T-\mu_{0}^T) \Sigma^{-1} (x-\mu_{0})
		+ \frac{1}{2}(x^T-\mu_1^T) \Sigma^{-1} (x-\mu_1) \\
	&= -\frac{1}{2}(x^T\Sigma^{-1}x -x^T\Sigma^{-1}\mu_{0}
			-\mu_{0}^T\Sigma^{-1}x + \mu_{0}^T\Sigma^{-1}\mu_{0})
			+ \frac{1}{2}(x^T\Sigma^{-1}x -x^T\Sigma^{-1}\mu_{1}
			-\mu_{1}^T\Sigma^{-1}x + \mu_{1}^T\Sigma^{-1}\mu_{1})
			\\
	&= -\frac{1}{2}(x^T\Sigma^{-1}\mu_{1} - x^T\Sigma^{-1}\mu_{0}
		+\mu_{1}^T\Sigma^{-1}x - \mu_{0}^T\Sigma^{-1}x
		+ \mu_{0}^T\Sigma^{-1}\mu_{0} - \mu_{1}^T\Sigma^{-1}\mu_{1})\\
	&= -\frac{1}{2}(x^T\Sigma^{-1}(\mu_{1}-\mu_{0})
		+ (\mu_{1}-\mu_{0})^T\Sigma^{-1}x
		+ \mu_{0}^T\Sigma^{-1}\mu_{0} - \mu_{1}^T\Sigma^{-1}\mu_{1}
		) \\
	&= -\frac{1}{2}(2(\mu_{1}-\mu_{0})^T\Sigma^{-1}x
		+ \mu_{0}^T\Sigma^{-1}\mu_{0} - \mu_{1}^T\Sigma^{-1}\mu_{1}
		) \\ 
	&= -(\mu_{1}-\mu_{0})^T\Sigma^{-1}x
		+ \frac{1}{2}(\mu_{1}^T\Sigma^{-1}\mu_{1} - \mu_{0}^T\Sigma^{-1}\mu_{0}
		)\\
	 &\\[100pt]
\end{align*}

\begin{align*}
p(y =1| x ; & \mc{H})\\
	&= \frac{1} {1+ 
		\exp\left(-(\mu_{1}-\mu_{0})^T\Sigma^{-1}x
		+ \frac{1}{2}(\mu_{1}^T\Sigma^{-1}\mu_{1} - \mu_{0}^T\Sigma^{-1}\mu_{0}) +\log((1-\phi)/\phi)
		\right) 
		}\\
	&= \frac{1} {1+ 
		\exp\left(-\left(\mu_{1}-\mu_{0})^T\Sigma^{-1}x
		+ \frac{1}{2}(\mu_{0}^T\Sigma^{-1}\mu_{0} - \mu_{1}^T\Sigma^{-1}\mu_{1}) -\log((1-\phi)/\phi)\right)
		\right) 
		}
	&
	 &\\[100pt]
\end{align*}

\begin{align*}
\theta^T &= (\mu_{1}-\mu_{0})^T\Sigma^{-1} \\
\theta_0 &= \frac{1}{2}(\mu_{0}^T\Sigma^{-1}\mu_{0} - \mu_{1}^T\Sigma^{-1}\mu_{1}) -\log((1-\phi)/\phi) 
	&
	 &\\[100pt]
\end{align*}

{\bf END PROOF}\\