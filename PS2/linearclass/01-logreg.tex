\item \subquestionpointswritten{6}

In lecture we saw the average empirical loss for logistic regression:
\begin{equation*}
	J(\theta)
	= -\frac{1}{\nexp} \sum_{i=1}^\nexp \left(y^{(i)}\log(h_{\theta}(x^{(i)}))
		+  (1 - y^{(i)})\log(1 - h_{\theta}(x^{(i)}))\right),
\end{equation*}
where $y^{(i)} \in \{0, 1\}$, $h_\theta(x) = g(\theta^T x)$ and $g(z) = 1 / (1 + e^{-z})$.

Find the Hessian $H$ of this function, and show that for any vector $z$, it holds true that
%
\begin{equation*}
    z^T H z \ge 0.
\end{equation*}
%
{\bf Hint:} You may want to start by showing that $\sum_i\sum_j z_i x_i x_j z_j = (x^Tz)^2 \geq 0$. Recall also that $g'(z) = g(z)(1-g(z))$.

{\bf Remark:} This is one of the standard ways of showing that the matrix $H$ is positive semi-definite, written ``$H \succeq 0$.''  This implies that $J$ is convex, and has no local minima other than the global one. If you have some other way of showing $H \succeq 0$, you're also welcome to use your method instead of the one above.\clearpage

{\bf BEGIN PROOF HERE}\\

Since $g'(z) = g(z)(1-g(z))$ and $h(x) = g(\theta^T x)$, it follows that $\partial h(x) / \partial \theta_k = h(x)(1 - h(x)) x_k$.

Letting $h_{\theta}(x^{(i)}) = g(\theta^T x^{(i)})
= 1/(1 + \exp(-\theta^T x^{(i)})$, we have\\

\begin{flalign*}
	\frac{\partial\log h_{\theta}(x^{(i)})}{\partial\theta_k} \\
	&= \frac{\partial\log g(\theta^T x^{(i)})}{\partial\theta_k} \\
	&= \frac{1}{g(\theta^T x^{(i)})}\frac{\partial g(\theta^T x^{(i)})}{\partial\theta_k} \\
	&= \frac{1}{g(\theta^T x^{(i)})}g(\theta^T x^{(i)})(1-g(\theta^T x^{(i)})) \frac{\partial (\theta^T x^{(i)})}{\partial\theta_k} \\
	 &= (1-g(\theta^T x^{(i)}))x_k^{(i)}
	  & & & & &\\[50pt]
	\frac{\partial\log(1 - h_{\theta}(x^{(i)}))}{\partial\theta_k} \\
	&= \frac{\partial\log(1 - g(\theta^T x^{(i)}))}{\partial\theta_k} \\
	&= \frac{-1}{1 - g(\theta^T x^{(i)})}\frac{\partial g(\theta^T x^{(i)})}{\partial\theta_k} \\
	&= \frac{-1}{1 - g(\theta^T x^{(i)})}g(\theta^T x^{(i)})(1-g(\theta^T x^{(i)}))\frac{\partial (\theta^T x^{(i)})}{\partial\theta_k} \\
	&= -g(\theta^T x^{(i)})x_k^{(i)}
	 & & & &
	\\[50pt]
\end{flalign*}

Substituting into our equation for $J(\theta)$, we have
%
\begin{flalign*}
	\frac{\partial J(\theta)}{\partial\theta_k} \\
	&= \frac{\partial}{\partial\theta_k}\frac{-1}{\nexp} \sum_{i=1}^\nexp \left(y^{(i)}\log(h_{\theta}(x^{(i)}))
		+  (1 - y^{(i)})\log(1 - h_{\theta}(x^{(i)}))\right) \\
	&=  \frac{-1}{\nexp} \sum_{i=1}^\nexp \left(y^{(i)}\frac{\partial}{\partial\theta_k}\log(h_{\theta}(x^{(i)}))
		+  (1 - y^{(i)})\frac{\partial}{\partial\theta_k}\log(1 - h_{\theta}(x^{(i)}))\right) \\
	&= \frac{-1}{\nexp} \sum_{i=1}^\nexp \left(y^{(i)} (1-g(\theta^T x^{(i)}))x_k^{(i)}
		+  (1 - y^{(i)})(-g(\theta^T x^{(i)})x_k^{(i)})\right) \\\\
	&= \frac{-1}{\nexp} \sum_{i=1}^\nexp \left(y^{(i)}x_k^{(i)} 
		- y^{(i)}g(\theta^T x^{(i)})x_k^{(i)} + y^{(i)}g(\theta^T x^{(i)})x_k^{(i)}
		- g(\theta^T x^{(i)})x_k^{(i)} \right) \\
	&= \frac{-1}{\nexp} \sum_{i=1}^\nexp \left(y^{(i)}x_k^{(i)} -g(\theta^T x^{(i)})x_k^{(i)} \right) \\
	&= \frac{-1}{\nexp} \sum_{i=1}^\nexp \left(y^{(i)}-g(\theta^T x^{(i)}) \right)x_k^{(i)}
	 & & &\\[50pt]
\end{flalign*}
%

Consequently, the $(k, l)$ entry of the Hessian is given by
%
\begin{flalign*}
	H_{kl} \\
	&= \frac{\partial^2 J(\theta)}{\partial\theta_k\partial\theta_l} \\
	&= \frac{\partial}{\partial\theta_l}\frac{\partial J(\theta)}{\partial\theta_k}\\
	&= \frac{\partial}{\partial\theta_l}
		\frac{-1}{\nexp} \sum_{i=1}^\nexp \left(y^{(i)}-g(\theta^T x^{(i)}) \right)x_k^{(i)} \\
	&= \frac{1}{\nexp} \sum_{i=1}^\nexp \frac{\partial}{\partial\theta_l} g(\theta^T x^{(i)})x_k^{(i)} \\
	&= \frac{1}{\nexp} \sum_{i=1}^\nexp g(\theta^T x^{(i)})(1-g(\theta^T x^{(i)}))x_k^{(i)} \frac{\partial}{\partial\theta_l}(\theta^T x^{(i)}) \\
	&= \frac{1}{\nexp} \sum_{i=1}^\nexp g(\theta^T x^{(i)})(1-g(\theta^T x^{(i)}))x_k^{(i)}x_l^{(i)}\\
	 & & & & &\\[50pt]
\end{flalign*}
%

Using the fact that $X_{ij} = x_i x_j$ if and only if $X = xx^T$, we have
%
\begin{flalign*}
	H \\
	&= \sum_k\sum_l H_{kl} \\
	&= \sum_k\sum_l \frac{1}{\nexp} \sum_{i=1}^\nexp g(\theta^T x^{(i)})(1-g(\theta^T x^{(i)}))x_k^{(i)}x_l^{(i)}
	 & & & & & & &\\[50pt]
\end{flalign*}

To prove that $H$ is positive semi-definite, show $z^T Hz \ge 0$ for all $z\in\Re^\di$.
%
\begin{flalign*}
	z^T H z \\
	& = \sum_k\sum_l \frac{1}{\nexp} \sum_{i=1}^\nexp g(\theta^T x^{(i)})(1-g(\theta^T x^{(i)}))z_k x_k^{(i)}x_l^{(i)} z_l \\
	&=  \frac{1}{\nexp} \sum_{i=1}^\nexp g(\theta^T x^{(i)})(1-g(\theta^T x^{(i)})) \sum_k\sum_l z_k x_k^{(i)}x_l^{(i)} z_l \\
	&= \frac{1}{\nexp} \sum_{i=1}^\nexp g(\theta^T x^{(i)})(1-g(\theta^T x^{(i)})) (x^T z)^2 
	& & & & & &\\[50pt]
\end{flalign*}
%
$g(\theta^T x^{(i)})$ is sigmoid with value between 0 and 1; and $(x^T z)^2 \geq 0$; therefore, $H \succeq 0$. \\\\

{\bf END PROOF}\\